\documentclass[12pt,a4paper]{article}
\usepackage[margin=2cm]{geometry}
\usepackage{xeCJK}
\usepackage{fontspec}
\setCJKmainfont{Noto Serif CJK TC}[Script=CJK]
\usepackage{amsmath,amssymb}
\usepackage{graphicx}
\usepackage{fancyhdr}
\setlength{\headheight}{14.5pt}
\addtolength{\topmargin}{-2.5pt}
\usepackage{hyperref}
\usepackage{listings}
\usepackage{enumitem}
\usepackage{titlesec}
\usepackage{caption}
\usepackage{indentfirst}
\setlength{\parindent}{2em}
\pagestyle{fancy}
\fancyhf{}
\cfoot{\thepage}
\linespread{1.3}

\title{資料結構HW2-Heap}
\author{B12508026戴偉璿}
\date{\today}

\begin{document}

\maketitle

\lhead{資料結構HW2-Heap}
\rhead{B12508026戴偉璿}

\newpage

\section{程式內容解析}
\subsection{前處理}

\subsubsection{輸出檔案}
首先使用\texttt{init()}函數宣告一個\texttt{ofstream}物件以清空\texttt{output}檔案,
接下來宣告一個\texttt{append}模式的\texttt{ofstream}物件\texttt{fout}以便於後續寫入排程結果。

\subsubsection{輸入檔案}
接著使用\texttt{read\_data()}函數,調用\texttt{ifstream}物件讀取\texttt{data}檔案,
這邊我把.csv檔案換成以空格分隔的純文字檔案方便處理

\subsubsection{型態別}
由於一個病人有三個數值(id,抵達時間,優先度)要紀錄,因此使用\texttt{struct}來定義一個\texttt{patient}結構體,方便處理

\subsubsection{檢查}
設計了一個\texttt{check()}函數來檢查是否有正確輸入,避免讀取到一堆亂碼或是錯誤的數值,
這個函數會輸出所有輸入的資料在終端機,由於我們處理過後的資料是直接使用檔案流輸出到檔案,因此不影響結果。

\subsection{Heap}
\subsubsection{Heap結構體}
Heap使用一個先前宣告過的\texttt{patient}結構體陣列來儲存病人資料,
並且使用一個整數變數\texttt{pt}來紀錄目前Heap的大小,\texttt{pt}的數值本身也代表這個Heap最後一個元素的index。

\subsubsection{插入}
插入到Heap的最後面,接下來再一一與他的父節點比較,
如果比父節點大就交換位置,直到不需要交換為止。

使用了位元運算計算父節點的index,
這樣可以節省一些運算時間。

\subsubsection{頂端元素}
取得Heap的頂端元素就是取得\texttt{hp[1]}

\subsubsection{刪除頂端元素}
通常取得頂端元素之後會將其刪除,
刪除頂端元素就是將最後一個元素放到頂端,將\texttt{pt}減一,
接下來再一一與他的子節點比較,
如果比子節點小就交換位置,直到不需要交換為止。

\subsubsection{是否為空}
判斷Heap是否為空就是判斷\texttt{pt}是否為0。

\subsubsection{除錯}
這個結構體提供了一個\texttt{show()}函數,
可以列印出Heap的內容,方便除錯。

\subsection{排程}
排程的過程比較麻煩,我們維護一個\texttt{current\_time}變數來紀錄目前的時間,
每次從Heap中取出頂端元素(也就是目前優先度最高的病人)之後,
就將時間增加他需要的治療時間,
接著將這個病人寫入\texttt{output}檔案中。
接下來再將所有抵達時間小於等於目前時間的病人加入Heap中,
然後再從Heap中取出頂端元素,重複這個過程直到Heap為空。
如果Heap為空但是還有病人沒有處理完,
就將時間增加到下一個病人抵達的時間,迴圈會自動判斷並姐將這個病人加入Heap中。

這一部份要非常小心,因為如果一口氣將所有小於當前時間的病人丟到Heap中,
然後一口氣處理完之後才加入下一組病人,可能會導致處理過程中有優先度更高的病人到來。

\section{排程結果(如output檔案)}

\begin{center}
\begin{tabular}{|c|c|c|}
\hline
Patient ID & Time & Priority \\
\hline
1 & 0$\sim$3 & 3 \\
2 & 3$\sim$6 & 3 \\
3 & 6$\sim$9 & 3 \\
5 & 9$\sim$14 & 2 \\
4 & 14$\sim$17 & 3 \\
8 & 17$\sim$22 & 2 \\
9 & 22$\sim$29 & 1 \\
13 & 29$\sim$34 & 2 \\
15 & 34$\sim$39 & 2 \\
17 & 39$\sim$46 & 1 \\
20 & 46$\sim$51 & 2 \\
23 & 51$\sim$56 & 2 \\
6 & 56$\sim$59 & 3 \\
26 & 59$\sim$64 & 2 \\
7 & 64$\sim$67 & 3 \\
29 & 67$\sim$74 & 1 \\
10 & 74$\sim$77 & 3 \\
11 & 77$\sim$80 & 3 \\
12 & 80$\sim$83 & 3 \\
14 & 83$\sim$86 & 3 \\
16 & 86$\sim$89 & 3 \\
18 & 89$\sim$92 & 3 \\
19 & 92$\sim$95 & 3 \\
21 & 95$\sim$98 & 3 \\
22 & 98$\sim$101 & 3 \\
24 & 101$\sim$104 & 3 \\
25 & 104$\sim$107 & 3 \\
27 & 107$\sim$110 & 3 \\
28 & 110$\sim$113 & 3 \\
30 & 113$\sim$116 & 3 \\
\hline
\end{tabular}
\end{center}



\end{document}