\documentclass[12pt,a4paper]{article}
\usepackage[margin=2cm]{geometry}
\usepackage{xeCJK}
\usepackage{fontspec}
\setCJKmainfont{Noto Serif CJK TC}[Script=CJK]
\usepackage{amsmath,amssymb}
\usepackage{graphicx}
\usepackage{fancyhdr}
\setlength{\headheight}{14.5pt}
\addtolength{\topmargin}{-2.5pt}
\usepackage{hyperref}
\usepackage{xcolor}
\usepackage{listings}
\usepackage{enumitem}
\usepackage{titlesec}
\usepackage{caption}
\usepackage{indentfirst}
\setlength{\parindent}{2em}
\pagestyle{fancy}
\fancyhf{}
\cfoot{\thepage}
\linespread{1.3}

\title{HW2 Hashing}
\author{B12508026戴偉璿}
\date{\today}

\lstset{
    basicstyle=\ttfamily\footnotesize,  % 字型與大小
    keywordstyle=\color{blue},
    commentstyle=\color{gray},
    stringstyle=\color{orange},
    numbers=left,                       % 行號在左側
    numberstyle=\tiny\color{gray},
    stepnumber=1,                       % 每行都顯示行號
    numbersep=5pt,
    backgroundcolor=\color{white},
    frame=single,                       % 加上框線
    breaklines=true,                    % 長行自動換行
    tabsize=2,
    language=C++                     % 可以換成 \texttt{C++}, \texttt{Java}, etc.
}


\begin{document}

\maketitle

\lhead{HW2 Hashing}
\rhead{B12508026戴偉璿}
\newpage

\section*{Part A}

我使用的hash方法是middle square method,這個方法的原理是將數字平方後取中間的四位數作為hash值。
這可以避免資料連續時有過多的碰撞。當碰撞發生後,我使用的probing方法是
混合quadratic probing以及linear probing,
在使用quadratic probing發生兩次碰撞(secondary clustering)之後就改為使用linear probing,
這樣可以避免產生過大的secondary clustering,但如果在相同的部位發生太多次的碰撞,
仍然可能會有primary clustering的問題。

發生secondary clustering的位置在id=18(資料11508011、12508729會有相同的hash值)、
id=40(資料11508863、11508599會有相同的hash值)。在table\_size為59的時候僅發生這兩次碰撞。

計算方法為C++程式

\section*{Part B}
這個方法在所有的資料中僅發生了兩次的碰撞,已經算是極低,難以再想出新的優化方法。

\section*{程式碼}
\begin{lstlisting}
#include<bits/stdc++.h>
#define int long long
#define eb emplace_back
using namespace std;

const int N=1000000, table_size=59, max_probing_times=2;

int arr[25]={10102109, 10106402, 10106918, 12508729, 12508629, 12508765,
            12508068, 12508705, 12508842, 11508011, 11508817, 11508388,
            11508189, 11508331, 11508675, 11508521, 11508287, 11508863,
            11508979, 11508532, 11508035, 11508599, 10613285, 13945978, 12945157};

int hash_table[table_size];

//i:原始資料的位置, id:hash table中的位置, x:探查的次數
inline void quandratic_probing(int i, int id, int x){
    while(x>max_probing_times){
        if(id>=table_size)id%=table_size;
        if(!hash_table[id]){
            hash_table[id]=arr[i];
            return;
        }
        ++id;
    }

    if(!hash_table[id]){
        hash_table[id]=arr[i];
        return;
    }
    id=(id+x*x)%table_size;
    quandratic_probing(i, id, x+1);
}

int32_t main(){
    for(int i=0;i<25;++i){
        int id=((arr[i]*arr[i]/N)%N)%table_size;
        if(hash_table[id])quandratic_probing(i, id, 1);
        else hash_table[id]=arr[i];
    }

    for(int i=0;i<table_size;++i){
        cout<<i<<":"<<hash_table[i]<<'\n';
    }

}
/*
作者:戴偉璿
日期:2025/05/01
說明:

使用的方法為mid-square,處理collision的方法為quandratic

為了避免一直找不到空位導致遞迴次數過多而stack overflow,因此設置一個最大探查次數(max_probing_times),一旦超過這個數值則進行linear probing,由於table size大於資料數量,因此每個資料都能找到位置
*/
\end{lstlisting}

\end{document}