\documentclass[11pt,a4paper]{article}

\usepackage{indentfirst}
\usepackage{amssymb}
\usepackage{subcaption}
\usepackage{graphicx}
\usepackage{longtable}
\usepackage{fancyhdr}
\usepackage{xeCJK}
\usepackage{amsmath}
\usepackage{amssymb}
\usepackage{ulem}
\usepackage{xcolor}
\usepackage{fancyvrb}
\usepackage{listings}
\usepackage{soul}
\usepackage{hyperref}
	\lstdefinestyle{C}{
   		language=C, 
   		basicstyle=\ttfamily\bfseries,
    	numbers=left, 
    	numbersep=5pt,
    	tabsize=4,
    	frame=single,
   	 	commentstyle=\itshape\color{brown},
    	keywordstyle=\bfseries\color{blue},
   	 	deletekeywords={define},
    	morekeywords={NULL,bool}
	}	

\setCJKmainfont{Noto Sans Mono CJK TC}
 
\voffset -20pt
\textwidth 410pt
\textheight 650pt
\oddsidemargin 20pt
\newcommand{\XOR}{\otimes}
\linespread{1.2}\selectfont

\pagestyle{fancy}
\lhead{第三屆卓越盃資訊學科能力競賽}

\begin{document}

\begin{center}
\section*{A. 開門問題}
\end{center}

\section*{Description}

你面前有$n$扇門,而你,則是一個無情的開門機器人,會無止盡的將門給打開。

每扇門都有上鎖,而且需要特殊的鑰匙才能夠打開,編號為$1$的門需要編號$1$的鑰匙,編號$2$的門需要編號$2$的鑰匙...以此類推。當你打開第$i$扇門之後,你會獲得編號為$k_i$的鑰匙,接下來你就會去開啟編號為$k_i$的門。

現在給你編號為$s$的鑰匙,請問你最多能開啟幾扇門?
	
\section*{Input}

第一行為兩個整數$n,s$,代表門的數量和初始鑰匙的編號

第二行為$n$個整數$k_1\sim k_n$

各變數範圍如下:
\begin{itemize}
    \item $1 \le n\le 10^5$
    \item $1\le s\le n$
    \item $\forall k_i\in [0,n]$
\end{itemize}\

\section*{Output}

請輸出你最多能開啟幾扇門

\section*{Sample 1}
\begin{longtable}[!h]{|p{0.5\textwidth}|p{0.5\textwidth}|}
\hline
\textbf {Input}	& \textbf {Output} \\
\hline
\parbox[t]{0.5\textwidth} % sample 1
{ \tt
% input
5 3\\
5 3 4 1 1\\
} &
\parbox[t]{0.5\textwidth}
{ \tt
%output
4\\
} \\
\hline
\end{longtable}

\section*{Sample 2}
\begin{longtable}[!h]{|p{0.5\textwidth}|p{0.5\textwidth}|}
\hline
\textbf {Input}	& \textbf {Output} \\
\hline
\parbox[t]{0.5\textwidth} % sample 2
{ \tt
% input
3 1\\
2 3 0\\
} &
\parbox[t]{0.5\textwidth}
{ \tt
%output
3\\
} \\
\hline
\end{longtable}

\section*{Sample 3}
\begin{longtable}[!h]{|p{0.5\textwidth}|p{0.5\textwidth}|}
\hline
\textbf {Input}	& \textbf {Output} \\
\hline
\parbox[t]{0.5\textwidth} % sample 3
{ \tt
% input
8 3\\
1 4 2 8 5 7 1 4\\
} &
\parbox[t]{0.5\textwidth}
{ \tt
%output
4\\
} \\
\hline
\end{longtable}

\section*{配分}

在一個子任務的「測試資料範圍」的敘述中,如果存在沒有提到範圍的變數,則此變數的範圍為 Input 所描述的範圍。

\begin{center}
 \begin{tabular}{||c c c||} 
 \hline
 子任務編號 & 子任務配分 & 測試資料範圍 \\  
 \hline
 \hline
 1 & 5\% & 範例測資 \\ 
 \hline
 2 & 20\% & $n=3$ \\
 \hline 
 3 & 75\% & $n\le 10^5$ \\
 \hline
\end{tabular}
\end{center}

\section*{Note}

\begin{itemize}
    \item 第一筆範例測資開門的順序:$3\rightarrow 4\rightarrow 1\rightarrow 5$
    \item $k_i=0$代表該扇門後沒有鑰匙

\end{itemize}\

\section*{Source}

改編自Educational Codeforces Round 132(Div.2) problem A



\end{document}