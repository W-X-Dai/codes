\documentclass[11pt,a4paper]{article}

\usepackage{indentfirst}
\usepackage{amssymb}
\usepackage{subcaption}
\usepackage{graphicx}
\usepackage{longtable}
\usepackage{fancyhdr}
\usepackage{xeCJK}
\usepackage{amsmath}
\usepackage{amssymb}
\usepackage{ulem}
\usepackage{xcolor}
\usepackage{fancyvrb}
\usepackage{listings}
\usepackage{soul}
\usepackage{hyperref}
	\lstdefinestyle{C}{
   		language=C, 
   		basicstyle=\ttfamily\bfseries,
    	numbers=left, 
    	numbersep=5pt,
    	tabsize=4,
    	frame=single,
   	 	commentstyle=\itshape\color{brown},
    	keywordstyle=\bfseries\color{blue},
   	 	deletekeywords={define},
    	morekeywords={NULL,bool}
	}	

\setCJKmainfont{Noto Sans Mono CJK TC}
 
\voffset -20pt
\textwidth 410pt
\textheight 650pt
\oddsidemargin 20pt
\newcommand{\XOR}{\otimes}
\linespread{1.2}\selectfont

\pagestyle{fancy}
\lhead{賭博遊戲}

\begin{document}

\begin{center}
\section*{賭博遊戲}
\end{center}

\section*{Description}

Marian 在賭場,賭場的歸則是這樣的:

在每一輪開始之前,玩家會從 $1$ 到 $10^9$之間押注一個數字. 接下來, 一個有 $10^9$ 面的骰子(~~相當詭異~~)會骰出一個從 $1$ 到 $10^9$ 的數字。如果玩家押注到了正確的數字的話他的獎金會翻倍,否則會減半

Marian 預測出了接下來$N$輪骰子會骰出的數字 $x_i\sim x_n$,他決定在第$l$到$r$輪中押注相同的數字$a$,請告訴他他該在哪些輪押注哪個數字可以贏得最多金額。

由於他用的是賭場的特殊籌碼,因此不需要顧慮整數的問題,諸如以下:$\cfrac{1}{1024}, \cfrac{1}{128}, \cfrac{1}{2}, 1, 2, 4\cdots$的金額都是可能出現的
	
\section*{Input}

第一行為一個整數$N$,接下來有$N$個數字$x_1\sim x_N$,第$i$個數字$x_i$代表第$i$輪骰子所骰出的值
\\
\\
各變數範圍如下:
\begin{itemize}
    \item $1 \le N\le 2\times 10^5$
    \item $1\le a,x_i\le 10^9$
    \item $1\le l\le r\le N$
\end{itemize}\

\section*{Output}

請依序輸出三個數$a,l,r$,代表在$l\sim r$輪時押注$a$可以贏得最多獎金

如果有多組解答,輸出任意一組即可。

\section*{Sample 1}
\begin{longtable}[!h]{|p{0.5\textwidth}|p{0.5\textwidth}|}
\hline
\textbf {Input}	& \textbf {Output} \\
\hline
\parbox[t]{0.5\textwidth} % sample 1
{ \tt
% input
5\\
4 4 3 4 4\\
} &
\parbox[t]{0.5\textwidth}
{ \tt
%output
4 1 5\\
} \\
\hline
\end{longtable}

\section*{Sample 2}
\begin{longtable}[!h]{|p{0.5\textwidth}|p{0.5\textwidth}|}
\hline
\textbf {Input}	& \textbf {Output} \\
\hline
\parbox[t]{0.5\textwidth} % sample 2
{ \tt
% input
1\\
1000000000\\
} &
\parbox[t]{0.5\textwidth}
{ \tt
%output
1000000000 1 1\\
} \\
\hline
\end{longtable}

\section*{Sample 3}
\begin{longtable}[!h]{|p{0.5\textwidth}|p{0.5\textwidth}|}
\hline
\textbf {Input}	& \textbf {Output} \\
\hline
\parbox[t]{0.5\textwidth} % sample 3
{ \tt
% input
10\\
8 8 8 9 9 6 6 9 6 6\\
} &
\parbox[t]{0.5\textwidth}
{ \tt
%output
6 6 10\\
} \\
\hline
\end{longtable}

\section*{Subtasks}

在一個子任務的「測試資料範圍」的敘述中,如果存在沒有提到範圍的變數,則此變數的範圍為 Input 所描述的範圍。

\begin{center}
 \begin{tabular}{||c c c||} 
 \hline
 子任務編號 & 子任務配分 & 測試資料範圍 \\  
 \hline
 \hline
 1 & 0\% & 範例測資 \\ 
 \hline
 %O(n^4)
 2 & 5\% & $N\le 100$\\
 \hline 
 % O(n^3)
 3 & 10\% & $N\le 500$ \\
 \hline
 % O(xn^2)
 4 & 10\% & $N\le 1000,\forall x_i\le 100$ \\
 \hline
 5 & 35\% & $N\le 2\times 10^5,\forall x_i\le 10^5$ \\
 \hline
 6 & 40\% & 無額外限制 \\
 \hline

\end{tabular}
\end{center}

\end{document}
